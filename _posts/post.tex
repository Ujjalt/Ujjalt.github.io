
\documentclass[11pt]{article}
\usepackage[a4paper, margin=1in]{geometry}
\usepackage{amsmath, amssymb}
\usepackage{hyperref}
\usepackage{xcolor}
\usepackage{graphicx}
\usepackage{tcolorbox}
\usepackage{setspace}
\usepackage{parskip}

\definecolor{myred}{RGB}{200,0,0}
\newcommand{\highlight}[1]{\textcolor{myred}{\textbf{#1}}}

\setstretch{1.15}

\title{\LARGE \highlight{Thomas Harriot’s Theorem and Angular Excess in Spherical Geometry}}
\author{\large Ujjal Timshina}
\date{}

\begin{document}

\maketitle

\vspace{1em}

In this note, we explore a fundamental result in \highlight{non-Euclidean geometry}, particularly \highlight{spherical geometry}, where the sum of the angles in a triangle \highlight{exceeds \( \pi \)} (180°). This deviation from Euclidean intuition leads to the concept of \emph{angular excess}, culminating in a theorem due to \highlight{Thomas Harriot}.

\bigskip

\highlight{Parallel Axiom and Non-Euclidean Geometry.} The Euclidean \emph{Parallel Postulate} is equivalent to:

\begin{quote}
The sum of the interior angles of any triangle is exactly \( \pi \).
\end{quote}

When this postulate is removed, two types of geometries emerge:
\begin{itemize}
    \item \highlight{Hyperbolic geometry} — angle sum \textit{less than} \( \pi \),
    \item \highlight{Spherical geometry} — angle sum \textit{greater than} \( \pi \).
\end{itemize}

These lead to two distinct conclusions:
\begin{itemize}
    \item More than one parallel line: hyperbolic geometry,
    \item No parallel lines: spherical geometry.
\end{itemize}

\bigskip

\highlight{Angular Excess in Spherical Geometry.}  
Let the radius of the sphere be \( R \). Then the total surface area is:
\[
\text{Area} = 4\pi R^2.
\]

Spherical triangles are formed by arcs of \emph{great circles} (i.e., geodesics on the sphere).

\medskip

\highlight{Example.} Consider a triangle formed by joining the North Pole to two points on the equator separated by angle \( \theta \). This triangle has angles:
\[
\theta, \quad \frac{\pi}{2}, \quad \frac{\pi}{2}
\]
so its angle sum is:
\[
\text{Angle sum} = \theta + \frac{\pi}{2} + \frac{\pi}{2} = \theta + \pi,
\]
and thus the angular excess is:
\[
E = (\text{angle sum}) - \pi = \theta.
\]

This triangle occupies a fraction \( \frac{\theta}{2\pi} \) of the northern hemisphere, whose area is \( 2\pi R^2 \), so:
\[
A = \frac{\theta}{2\pi} \cdot 2\pi R^2 = \theta R^2,
\]
which implies:
\[
E = \frac{A}{R^2}.
\]

\bigskip

\highlight{Thomas Harriot’s Theorem (1603).}  
This identity holds for all spherical triangles. Let a triangle \( \Delta \) have interior angles \( \alpha, \beta, \gamma \). Then:
\[
E = \alpha + \beta + \gamma - \pi
\]
and Harriot proved:
\[
A(\Delta) = R^2 E.
\]

\begin{tcolorbox}[colback=white, colframe=myred, title=\textbf{Harriot’s Formula}, title filled=true]
\[
\boxed{A(\Delta) = R^2(\alpha + \beta + \gamma - \pi)}
\]
\end{tcolorbox}

Harriot’s geometric argument compares the areas of wedges formed by extending great-circle sides and summing their proportions over the sphere.

\bigskip

\highlight{Conclusion.}  
On a sphere of radius \( R \), every triangle satisfies:
\[
A = R^2 \cdot \left( (\alpha + \beta + \gamma) - \pi \right) = R^2 E.
\]

This identity links \highlight{curvature} with \highlight{angle geometry}, and foreshadows the \emph{Gauss–Bonnet Theorem}. It also appears in \emph{general relativity}, where curvature governs the shape of space and time.

\bigskip

\highlight{Further Reading and Tools:}
\begin{itemize}
    \item �� \href{https://www.youtube.com/watch?v=Y8VgvoEx7HY}{\textbf{Complete geometric proof on YouTube}}
    \item �� \href{https://www.greatcirclemap.com/}{\textbf{Great Circle Map — Visualize real geodesics on the Earth’s sphere}}
\end{itemize}

\end{document}
1
